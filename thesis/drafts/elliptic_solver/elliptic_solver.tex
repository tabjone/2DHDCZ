\documentclass{article}
\usepackage{amsmath, amsthm, amssymb}
\usepackage{natbib}

% Uncomment to use Times New Roman font
% \usepackage{times}

% Margins
\usepackage[margin=1in]{geometry}

% Header/Footer
\usepackage{fancyhdr}
\pagestyle{fancy}
\fancyhf{}
\lhead{Anelastic approximation draft}
\rhead{\today}
\cfoot{Page \thepage}

\begin{document}

\section{Elliptic solver}

Second order elliptic linear partial differential equation. Elliptic because discriminant is -4.


Elliptic equation = equilibrium problem.

Inverse of matrix is inneficient in terms of FLOPs. A will be a symmetric square matrix.
Gaussian elimination: Row reduction is called forward substitution in the litterature.

Vanilla Dirichlet boundary conditions = 0. General = any number.

\textit{Momentum equation}
\begin{equation}\label{eq:momentum_full}
    \rho_0\left[\frac{\partial\mathbf{v}}{\partial t}+(\mathbf{v}\cdot\nabla)\mathbf{v}\right]=-\nabla p_1 + \rho_1\mathbf{g}.
\end{equation}

We take the divergence of the momentum equation

\begin{equation}
    \nabla\cdot \left[\rho_0\frac{\partial\mathbf{v}}{\partial t}+\rho_0(\mathbf{v}\cdot\nabla)\mathbf{v}\right]=-\nabla^2 p_1 + \nabla\cdot\left(\rho_1\mathbf{g}\right).
\end{equation}

\begin{equation}
     \nabla\cdot\frac{\partial\left(\rho_0\mathbf{v}\right)}{\partial t}+\nabla\cdot\rho_0(\mathbf{v}\cdot\nabla)\mathbf{v}=-\nabla^2 p_1 - \partial_z\left(\rho_1 g\right).
\end{equation}

Using Clairaut's theorem on the first term

\begin{equation}
    \frac{\partial\left[\nabla\cdot\left(\rho_0\mathbf{v}\right)\right]}{\partial t}+\nabla\cdot\rho_0(\mathbf{v}\cdot\nabla)\mathbf{v}=-\nabla^2 p_1 - \partial_z\left(\rho_1 g\right).
\end{equation}

Then the first term is zero by the continuity equation, we solve for $\nabla^2 p_1$

\begin{equation}
    \nabla^2 p_1= - \partial_z\left(\rho_1 g\right) - \nabla\cdot\rho_0(\mathbf{v}\cdot\nabla)\mathbf{v}.
\end{equation}

Earlier we found that 

\begin{equation}
    (\mathbf{v}\cdot\nabla)\mathbf{v} = (v_x\partial_x+ v_z\partial_z)v_x\mathbf{\hat{i}}+(v_x\partial_x+ v_z\partial_z)v_z\mathbf{\hat{k}}.
\end{equation}

Inserting this

\begin{equation}
    \nabla^2 p_1= - \partial_z\left(\rho_1 g\right) - \nabla\cdot\rho_0\left[(v_x\partial_x+ v_z\partial_z)v_x\mathbf{\hat{i}}+(v_x\partial_x+ v_z\partial_z)v_z\mathbf{\hat{k}} \right].
\end{equation}

\begin{equation}
    \nabla^2 p_1= - \partial_z\left(\rho_1 g\right) - \nabla\cdot\left[(\rho_0v_x\partial_x+ \rho_0v_z\partial_z)v_x\mathbf{\hat{i}}+(\rho_0v_x\partial_x+ \rho_0v_z\partial_z)v_z\mathbf{\hat{k}} \right].
\end{equation}

\begin{equation}
    \nabla^2 p_1= - \partial_z\left(\rho_1 g\right) - \partial_x\left[\rho_0v_x\partial_xv_x+ \rho_0v_z\partial_zv_x\right]-\partial_z\left[\rho_0v_x\partial_xv_z + \rho_0v_z\partial_zv_z \right].
\end{equation}

\begin{align*}
    \nabla^2 p_1 = &- \partial_z\left(\rho_1 g\right) \\
    &-\partial_x(v_x)\partial_x(\rho_0v_x)-\rho_0v_x\partial^2_x v_x\\
    &-\partial_z(v_x)\partial_x(\rho_0v_z)-\rho_0v_z \partial_x\partial_z(v_x)\\
    &-\partial_x(v_z)\partial_z(\rho_0v_x)-\rho_0v_x\partial_z\partial_x(v_z)\\
    &-\partial_z(v_z)\partial_z(\rho_0v_z)-\rho_0v_z\partial^2_z v_z.
\end{align*}

By the continuity equation we can remove some terms.

\begin{align*}
    \nabla^2 p_1 = &- \partial_z\left(\rho_1 g\right) \\
    &-\rho_0v_x\partial^2_x v_x\\
    &-\partial_z(v_x)\partial_x(\rho_0v_z)-\rho_0v_z \partial_x\partial_z(v_x)\\
    &-\partial_x(v_z)\partial_z(\rho_0v_x)-\rho_0v_x\partial_z\partial_x(v_z)\\
    &-\rho_0v_z\partial^2_z v_z.
\end{align*}

By $\rho_0$ being constant in the x-direction we get

\begin{align*}
    \nabla^2 p_1 = &- \partial_z\left(\rho_1 g\right) \\
    &-v_x\partial^2_x (\rho_0 v_x)\\
    &-\partial_z(v_x)\partial_x(\rho_0v_z)-\rho_0v_z \partial_x\partial_z(v_x)\\
    &-\partial_x(v_z)\partial_z(\rho_0v_x)-\rho_0v_x\partial_z\partial_x(v_z)\\
    &-\rho_0v_z\partial^2_z v_z.
\end{align*}

And we again use the continuity equation

\begin{align*}
    \nabla^2 p_1 = &- \partial_z\left(\rho_1 g\right) \\
    &-\partial_z(v_x)\partial_x(\rho_0v_z)-\rho_0v_z \partial_x\partial_z(v_x)\\
    &-\partial_x(v_z)\partial_z(\rho_0v_x)-\rho_0v_x\partial_z\partial_x(v_z)\\
    &-\rho_0v_z\partial^2_z v_z.
\end{align*}


\begin{equation}
    \nabla^2 p_1 = - \partial_z\left(\rho_1 g\right)
    -\rho_0\left[v_z \partial_x\partial_z(v_x)+v_x\partial_z\partial_x(v_z)+v_z\partial^2_z v_z \right]
    -\partial_x(v_z)\partial_z(\rho_0v_x)
    -\partial_z(v_x)\partial_x(\rho_0v_z)
\end{equation}

We again use that $\rho_0$ is constant in the x-direction

\begin{equation}
    \nabla^2 p_1 = - \partial_z\left(\rho_1 g\right)
    -\rho_0\left[v_z \partial_x\partial_z(v_x)+v_x\partial_z\partial_x(v_z)+v_z\partial^2_z v_z+\partial_z(v_x)\partial_x(v_z) \right]
    -\partial_x(v_z)\partial_z(\rho_0v_x)
\end{equation}

This can be simplified to

\begin{equation}
    \nabla^2 p_1 = - \partial_z\left(\rho_1 g\right)
    -\rho_0\left[v_z \partial_x\partial_z(v_x)+v_x\partial_z\partial_x(v_z)+v_z\partial^2_z v_z+\partial_z(v_x)\partial_x(v_z) \right]
    -\partial_x(v_z)\partial_z(\rho_0v_x)
\end{equation}

Simplifying the bracket term

\begin{equation}
    \nabla^2 p_1 = - \partial_z\left(\rho_1 g\right)
    -\rho_0\left[\partial_x\partial_z(v_x v_z)+v_z\partial^2_z v_z+\partial_z(v_x)\partial_x(v_z) \right]
    -\partial_x(v_z)\partial_z(\rho_0v_x)
\end{equation}

Expanding the last term

\begin{equation}
    \nabla^2 p_1 = - \partial_z\left(\rho_1 g\right)
    -\rho_0\left[\partial_x\partial_z(v_x v_z)+v_z\partial^2_z v_z+\partial_z(v_x)\partial_x(v_z) \right]
    -\rho_0\partial_x(v_z)\partial_z(v_x)-v_x \partial_x(v_z)\partial_z(\rho_0)
\end{equation}

\begin{equation}
    \nabla^2 p_1 = - \partial_z\left(\rho_1 g\right)
    -\rho_0\left[\partial_x\partial_z(v_x v_z)+v_z\partial^2_z v_z+\partial_z(v_x)\partial_x(v_z) +\partial_x(v_z)\partial_z(v_x)\right]
    -v_x \partial_x(v_z)\partial_z(\rho_0)
\end{equation}

\begin{equation}
    \nabla^2 p_1 = - \partial_z\left(\rho_1 g\right)
    -\rho_0\left[\partial_x\partial_z(v_x v_z)+v_z\partial^2_z v_z+2\partial_z(v_x)\partial_x(v_z)\right]
    -v_x \partial_x(v_z)\partial_z(\rho_0)
\end{equation}

THERE MUST BE SOME WAY TO SIMPLIFY THIS BUT I CAN'T SEE IT!



\bibliographystyle{apalike}
\bibliography{initialization.bib}

\end{document}
