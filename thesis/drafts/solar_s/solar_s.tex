\documentclass{article}
\usepackage{amsmath, amsthm, amssymb}
\usepackage{natbib}

% Uncomment to use Times New Roman font
% \usepackage{times}

% Margins
\usepackage[margin=1in]{geometry}

% Header/Footer
\usepackage{fancyhdr}
\pagestyle{fancy}
\fancyhf{}
\lhead{Anelastic approximation draft}
\rhead{\today}
\cfoot{Page \thepage}

\begin{document}

\section{Solar S}

The gravitational acceleration of an object is given by Newtons law of gravitation as

\begin{equation}
    g(r)=\frac{GM(r)}{r^2},
\end{equation}
where $M(r)$ is the enclosed mass, $G=G = 6.6743\cdot10^8\ [\text{cm}^3/\text{gs}^2]$ is the gravitational constant and $r$ is the distance from the centre of mass. We rename this as $r\rightarrow r R_*$, where $R_*$ is the radius of the sun since the solar S data is in $r/R_*$. Then the gravitational law is

\begin{equation}
    g(r)=\frac{GM(r)}{R_*^2r^2}.
\end{equation}

The enclosed mass is
\begin{equation}
    M(r) = \int_0^r dm = \int_0^r =4\pi R_*^3 \int_0^r \rho(r')dV=\int_0^r \rho(r')r'^2 dr'
\end{equation}

We discretize this and use the trapezoidal method of integration and get
\begin{equation}
    M(r_{i+1}) = 4\pi R_*^3 \frac{1}{2}(\rho_{0,(i+1)}r_{i+1}^2-\rho_{0,(i)}r_{i}^2)(r_{i+1}-r_i).
\end{equation}

Inserting this in the equation for gravitational acceleration gives

\begin{equation}
    g(r_{i+1})=\frac{4\pi R_* G}{r_{i+1}^2}\frac{1}{2}(\rho_{i+1} r_{i+1}^2-\rho_i r_i^2)(r_{i+1}-r_{i})
\end{equation}

\end{document}