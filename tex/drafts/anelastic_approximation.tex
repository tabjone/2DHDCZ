\documentclass{article}
\usepackage{amsmath, amsthm, amssymb}
\usepackage{natbib}

% Uncomment to use Times New Roman font
% \usepackage{times}

% Margins
\usepackage[margin=1in]{geometry}

% Header/Footer
\usepackage{fancyhdr}
\pagestyle{fancy}
\fancyhf{}
\lhead{Anelastic approximation draft}
\rhead{\today}
\cfoot{Page \thepage}

\begin{document}

\section{Anelastic Approximation}
\subsection{Pertubation theory}
Do I need a subsection on this?
\subsection{Mixing length theory}
\subsection{Anelastic approximation}

FROM LANTZ:

Original anelastic approximation: \citep{1962JAtS...19..159C} or \citep{1962JAtS...19..173O}. In these papers only diffusionless fluids.\\

Next step: \citep{1969JAtS...26..448G}. "The anelastic approximation can be maintained by relaxing the condition of strict adiabaticity. In his view, the reference state can even be time dependent; the only restriction is that the Mach number must remain small."  "Used Gough's anelastic model to compute convection cell structures in time-dependent, pulsating stars, based on simplifed, few-mode planforms like those in the Boussinesq
calculations by Gough, Spiegel, \& Toomre (1975). Problem: dangerous to relax the isentropic condition, because anywhere the reference entropy gradient is strongly superadiabatic, a high Mach number Ñow could well be the result. In particular, an upper thermal boundary layer might become supersonically unstable."

"Making the di†usive heat flux proportional to the local entropy gradient is one way to achieve a zero-order isentropic state even in the presence of thermal di†usion. This is the approach taken by Gilman \& Glatzmaier (1981) and Glatzmaier (1984), who saw this type of heat flux as a viable model for the subgrid diffusion of entropy in numerical simulations. From this viewpoint, the anelastic approximation becomes a system of equations not simply for slow convective motions, but for slow and large convective motions superimposed on a sea of smaller scale turbulent motions. The approximation therefore involves two types of filtering : time-domain filtering to remove sound waves and spatial filtering to average over the small-scale turbulence."

"The lowest order isentropic condition is met by setting the polytropic index $m=1.5$, and a first-order convective instability is created by making m just slightly less than this value."


\section{Eqs from Yuhong Fan}

\begin{align}
    \nabla\cdot(\rho\mathbf{v})=0
\end{align}

\begin{align}
    \rho_0\left[\frac{\partial\mathbf{v}}{\partial t}+(\mathbf{v}\cdot\nabla)\mathbf{v}\right]=-\nabla p_1 + \rho_1\mathbf{g} + \frac{1}{4\pi} (\nabla\times\mathbf{B})\times\mathbf{B}+\nabla\cdot\mathbf{\Pi},
\end{align}

\begin{align}
    \rho_0 T_0 \left[\frac{\partial s_1}{\partial t} + (\mathbf{v}\cdot \nabla)(s_0+s_1) \right]
    = \nabla\cdot(K\rho_0T_0\nabla s_1) +\frac{1}{4\pi}\eta |\nabla\times\mathbf{B}|^2 + (\mathbf{\Pi}\cdot\nabla)\cdot\mathbf{v},
\end{align}

\begin{align}
    \nabla\cdot\mathbf{B}=0,
\end{align}

\begin{align}
    \frac{\partial\mathbf{B}}{\partial t} = \nabla\times(\mathbf{v}\times\mathbf{v})-\nabla\times(\eta\nabla\times\mathbf{B}),
\end{align}

\begin{align}
    \frac{\rho_1}{\rho_0} = \frac{p_1}{p_0} - \frac{T_1}{T_0},
\end{align}

\begin{align}
    \frac{s_1}{c_p} = \frac{T_1}{T_0} - \frac{\gamma-1}{\gamma}\frac{p_1}{p_0}.
\end{align}

$\mu,K,\eta$ is the dynamic viscosity, and thermal and magnetic diffusivity.


"Because of the divergence free condition of $\rho_0\mathbf{v}$ given in equation (11), one can take the divergence of the momentum equation, where the divergence of the $\rho_0(\partial\mathbf{v}/\partial t$ term vanishes, to obtain an elliptic equation for $p_1$ of the form: $\nabla^2 p_1=....$. One way to numerically maintain equation (11) is to solve this elliptic equation for $p_1$ at every time step before substituting it into the momentum equation for advancing the velocity (e.g. Fan 2008). Another well-known method to ensure equation (11) in anelastic MHD codes that use the spectral method is to express $\rho_0\mathbf{v}$ in terms of the curls of vector potentials and numerically advance the equations for the vector potentials (e.g. Glatzmaier 1984; Fan et al. 1999; Featherstone and Hindman 2016)."


\bibliographystyle{apalike}
\bibliography{refs_anelastic}

\end{document}
