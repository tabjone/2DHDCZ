\documentclass{article}
\usepackage{amsmath, amsthm, amssymb}
\usepackage{natbib}

% Uncomment to use Times New Roman font
% \usepackage{times}

% Margins
\usepackage[margin=1in]{geometry}

% Header/Footer
\usepackage{fancyhdr}
\pagestyle{fancy}
\fancyhf{}
\lhead{Anelastic approximation draft}
\rhead{\today}
\cfoot{Page \thepage}

\begin{document}

\section{Anelastic Approximation}
\subsection{Pertubation theory}
Do I need a subsection on this?
\subsection{Mixing length theory}
\subsection{Anelastic approximation}

FROM LANTZ:

Original anelastic approximation: \citep{1962JAtS...19..159C} or \citep{1962JAtS...19..173O}. In these papers only diffusionless fluids.\\

Next step: \citep{1969JAtS...26..448G}. "The anelastic approximation can be maintained by relaxing the condition of strict adiabaticity. In his view, the reference state can even be time dependent; the only restriction is that the Mach number must remain small."  "Used Gough's anelastic model to compute convection cell structures in time-dependent, pulsating stars, based on simplifed, few-mode planforms like those in the Boussinesq
calculations by Gough, Spiegel, \& Toomre (1975). Problem: dangerous to relax the isentropic condition, because anywhere the reference entropy gradient is strongly superadiabatic, a high Mach number Ñow could well be the result. In particular, an upper thermal boundary layer might become supersonically unstable."

"Making the di†usive heat flux proportional to the local entropy gradient is one way to achieve a zero-order isentropic state even in the presence of thermal di†usion. This is the approach taken by Gilman \& Glatzmaier (1981) and Glatzmaier (1984), who saw this type of heat flux as a viable model for the subgrid diffusion of entropy in numerical simulations. From this viewpoint, the anelastic approximation becomes a system of equations not simply for slow convective motions, but for slow and large convective motions superimposed on a sea of smaller scale turbulent motions. The approximation therefore involves two types of filtering : time-domain filtering to remove sound waves and spatial filtering to average over the small-scale turbulence."

"The lowest order isentropic condition is met by setting the polytropic index $m=1.5$, and a first-order convective instability is created by making m just slightly less than this value."


\section{Eqs from Yuhong Fan}

\begin{align}
    \nabla\cdot(\rho\mathbf{v})=0
\end{align}

\begin{align}
    \rho_0\left[\frac{\partial\mathbf{v}}{\partial t}+(\mathbf{v}\cdot\nabla)\mathbf{v}\right]=-\nabla p_1 + \rho_1\mathbf{g} + \frac{1}{4\pi} (\nabla\times\mathbf{B})\times\mathbf{B}+\nabla\cdot\mathbf{\Pi},
\end{align}

\begin{align}
    \rho_0 T_0 \left[\frac{\partial s_1}{\partial t} + (\mathbf{v}\cdot \nabla)(s_0+s_1) \right]
    = \nabla\cdot(K\rho_0T_0\nabla s_1) +\frac{1}{4\pi}\eta |\nabla\times\mathbf{B}|^2 + (\mathbf{\Pi}\cdot\nabla)\cdot\mathbf{v},
\end{align}

\begin{align}
    \nabla\cdot\mathbf{B}=0,
\end{align}

\begin{align}
    \frac{\partial\mathbf{B}}{\partial t} = \nabla\times(\mathbf{v}\times\mathbf{B})-\nabla\times(\eta\nabla\times\mathbf{B}),
\end{align}

\begin{align}
    \frac{\rho_1}{\rho_0} = \frac{p_1}{p_0} - \frac{T_1}{T_0},
\end{align}

\begin{align}
    \frac{s_1}{c_p} = \frac{T_1}{T_0} - \frac{\gamma-1}{\gamma}\frac{p_1}{p_0}.
\end{align}

$\mu,K,\eta$ is the dynamic viscosity, and thermal and magnetic diffusivity.


"Because of the divergence free condition of $\rho_0\mathbf{v}$ given in equation (11), one can take the divergence of the momentum equation, where the divergence of the $\rho_0(\partial\mathbf{v}/\partial t$ term vanishes, to obtain an elliptic equation for $p_1$ of the form: $\nabla^2 p_1=....$. One way to numerically maintain equation (11) is to solve this elliptic equation for $p_1$ at every time step before substituting it into the momentum equation for advancing the velocity (e.g. Fan 2008). Another well-known method to ensure equation (11) in anelastic MHD codes that use the spectral method is to express $\rho_0\mathbf{v}$ in terms of the curls of vector potentials and numerically advance the equations for the vector potentials (e.g. Glatzmaier 1984; Fan et al. 1999; Featherstone and Hindman 2016)."

\subsection{First step: Solve for d/dt parts}
Eq 2:
\begin{align}
    \frac{\partial\mathbf{v}}{\partial t}=\frac{1}{\rho_0}\left[-\nabla p_1 + \rho_1\mathbf{g} + \frac{1}{4\pi} (\nabla\times\mathbf{B})\times\mathbf{B}+\nabla\cdot\mathbf{\Pi}\right]-(\mathbf{v}\cdot\nabla)\mathbf{v}
\end{align}

Eq 3:

\begin{align}
    \frac{\partial s_1}{\partial t} 
    = \frac{1}{\rho_0T_0}\left[\nabla\cdot(K\rho_0T_0\nabla s_1) +\frac{1}{4\pi}\eta |\nabla\times\mathbf{B}|^2 + (\mathbf{\Pi}\cdot\nabla)\cdot\mathbf{v}\right]-(\mathbf{v}\cdot \nabla)(s_0+s_1)
\end{align}

Eq 5: Stays the same.

\subsection{Second step: Split directions}

Eq 8: First we calculate some substeps
\begin{align*}
    [(\mathbf{v}\cdot\nabla)\mathbf{v}]_i = \left(v_x\partial_x v_i + v_y\partial_y v_i + v_z\partial_z v_i \right), i=x,y,z.
\end{align*}

$(\nabla\times\mathbf{B})\times\mathbf{B}$ gives these components in the different directions

\begin{align*}
    x&:\ (\partial_z B_x-\partial_x B_z )B_z - (\partial_x B_y - \partial_y B_x )B_y \\
    y&:\ (\partial_x B_y - \partial_y B_x )B_x -(\partial_y B_z - \partial_z B_y )B_z\\
    z&:\ (\partial_y B_z - \partial_z B_y )B_y - (\partial_z B_x - \partial_x B_z )B_x
\end{align*}

Gir da

\begin{align}
    x&:\ \partial_t v_x =  \frac{1}{\rho_0}\left[-(\nabla p_1)_x + \frac{1}{4\pi} \left( (\partial_z B_x-\partial_x B_z )B_z - (\partial_x B_y - \partial_y B_x )B_y \right)+(\nabla\cdot\mathbf{\Pi})_x\right]-\left(v_x\partial_x v_x + v_y\partial_y v_x + v_z\partial_z v_x \right) \\
    y&:\ \partial_t v_x = \frac{1}{\rho_0}\left[-(\nabla p_1)_y  + \frac{1}{4\pi}\left( (\partial_x B_y - \partial_y B_x )B_x -(\partial_y B_z - \partial_z B_y )B_z \right)  +(\nabla\cdot\mathbf{\Pi})_y\right]- \left(v_x\partial_x v_y + v_y\partial_y v_y + v_z\partial_z v_y \right) \\
    z&:\ \partial_t v_z = \frac{1}{\rho_0}\left[-(\nabla p_1)_z - \rho_1 g + \frac{1}{4\pi}\left( (\partial_y B_z - \partial_z B_y )B_y - (\partial_z B_x - \partial_x B_z )B_x \right) +(\nabla\cdot\mathbf{\Pi})_z\right]-\left(v_x\partial_x v_z + v_y\partial_y v_z + v_z\partial_z v_z \right)
\end{align}

Now for eq 9, first substeps

\begin{align*}
    \nabla\cdot(K\rho_0T_0\nabla s_1) = \sum\limits_{i=x,y,z}\partial_i(K\rho_0T_0\partial_i s_1)
\end{align*}

\begin{align*}
    |\nabla\times\mathbf{B}|^2 = (\partial_y B_z-\partial_zB_y)^2+(\partial_zB_x-\partial_xB_z)^2+(\partial_xB_y-\partial_yB_x)^2
\end{align*}

\begin{align}
    \partial_t s_1 =& \frac{1}{\rho_0T_0}\left[\sum\limits_{i=x,y,z}\partial_i(K\rho_0T_0\partial_i s_1) +\frac{1}{4\pi}\eta \left( (\partial_y B_z-\partial_zB_y)^2+(\partial_zB_x-\partial_xB_z)^2+(\partial_xB_y-\partial_yB_x)^2 \right) + (\mathbf{\Pi}\cdot\nabla)\cdot\mathbf{v}\right]\nonumber\\
    &-\sum\limits_{i=x,y,z}v_i\partial_i(s_0+s_1)
\end{align}

Eq 5 substeps. $\nabla\times(\mathbf{v}\times\mathbf{B})$ gives the three components
\begin{align*}
    x&:\ \partial_y(v_xB_y-v_yB_x) - \partial_z(v_zB_x-v_xB_z) \\
    y&:\  \partial_z(v_yB_z-v_zB_y)-\partial_x(v_xB_y-v_yB_x) \\
    z&:\ \partial_x(v_zB_x-v_xB_z) - \partial_y(v_yB_z-v_zB_y)
\end{align*}

Now $\nabla\times(\eta\nabla\times\mathbf{B})$
\begin{align*}
    x&:\ \partial_y\eta(\partial_xB_y-\partial_yB_x)-\partial_z\eta(\partial_zB_x-\partial_xB_z) \\
    y&:\ \partial_z\eta(\partial_yB_z-\partial_zB_y)-\partial_x\eta(\partial_xB_y-\partial_yB_x) \\
    z&:\ \partial_x\eta(\partial_zB_x-\partial_xB_z)-\partial_y\eta(\partial_yB_z-\partial_zB_y)
\end{align*}

This gives three equations for the B-field
\begin{align}
\partial_t B_x &= \partial_y(v_xB_y-v_yB_x) - \partial_z(v_zB_x-v_xB_z) - \left[ \partial_y\eta(\partial_xB_y-\partial_yB_x)-\partial_z\eta(\partial_zB_x-\partial_xB_z) \right]\\
\partial_t B_y &= \partial_z(v_yB_z-v_zB_y)-\partial_x(v_xB_y-v_yB_x) - \left[ \partial_z\eta(\partial_yB_z-\partial_zB_y)-\partial_x\eta(\partial_xB_y-\partial_yB_x) \right]\\
\partial_t B_z &= \partial_x(v_zB_x-v_xB_z) - \partial_y(v_yB_z-v_zB_y) - \left[ \partial_x\eta(\partial_zB_x-\partial_xB_z)-\partial_y\eta(\partial_yB_z-\partial_zB_y) \right]
\end{align}

\subsection{Third step: Solving numerically, method}
1. Find $p_1$ trough one of the two given methods.\\
2. Solve all d/dt eqs.\\
2. Put $s_1$ and $p_1$ into eq 7 to find $T_1$.\\
3. Put everything into eq. 6 to find $\rho_1$.

\subsection{Fourth step: Discretization, RK, Up/down, central}
Up/Down derivatives marked with ud subscript. Central marked with c. Discretize in time for first order and rk.



\bibliographystyle{apalike}
\bibliography{refs_anelastic}

\end{document}
